\chapter{Conclusions \& Future Work}
\label{chap:Conclusions}


\section{Future Work}

Our design of \ac{SECH} 2 specifies only \ac{AES}-128-\ac{GCM} as the \ac{AEAD} cipher for servername encryption, since
it is specified that a \ac{TLS}-compliant
server must implement this cipher.
Since the ciphertext is sent in the first flight before any negotiation takes place this simplifies the design.
The design could be extended to allow other ciphers,
as well as connection-specific negotiation of the servername encryption cipher.
For example, the \ac{AEAD} cipher could be negotiated as the preferred cipher in the 
\var{cipher\_suites} list.

Our design of \ac{SECH} 2 uses a 24 byte inner random and 12 byte inner payload.
There are several avenues we could explore in order to expand the available bandwidth for the inner payload (the \ac{SNI} and \ac{ALPN}).
For one, rather than transmitting the 24 byte inner random the inner random could be derived from the shared secret.
For split-mode, this derived inner random would need to be embedded in the \ac{CHI}.
Since the \ac{AEAD} \nonce is not needed by the backend server we could put part of the inner random in the first 12 bytes of the \ac{CHI}'s \var{random}.
More of the inner random could go in the \ac{SNI} extension data field, rather than transmitting a string of 0s, and finally,
the backend server does not need the \ac{AEAD} \ac{MAC}, so these 16 bytes could be used for the inner random.
With these changes we can increase the size of both the inner random {\em and} the \ac{SECH} payload.

For our design of \ac{SECH} 2 we abandoned the goal of offering
split mode in order to include a mitigation
against a \ac{HRR} hijacking attack that could break \ac{PC}.
We have not ruled out the possibility of both mitigating the
\ac{HRR} hijacking attack {\em and} offering split-mode.
This problem is a direction for future research.

More ambitiously, however, it would be ideal to find
a mitigation against the \ac{HRR} hijacking problem that
does not involve excessive cover messages and bandwidth.

% This chapter should summarize the work presented in the dissertation and discuss the conclusions that can be drawn from the work and the results presented in chapter~\ref{chap:Evaluation}.


% \section{Future Work}

% The section may present a list of items that were beyond the scope of the dissertation.