\chapter{Conclusions \& Future Work}
\label{chap:Conclusions}
The 64 bytes of cover provided by
the \var{random} and \var{legacy\_session\_id}
appear to be sufficient to implement both \ac{AEAD}-based
and \ac{HPKE}-based variants of the \ac{ECH} protocol,
each with either connection-level or channel-level \ac{PC}.

The requirement of not sticking out under an active-attacker threat model results
in the necessity of both client and server sending more messages than are necessary for \ac{ECH} in failure cases,
which makes our designs for \ac{SECH} less appealing.

There remains a risk, also, that the use of ciphertext in
the place of the \var{random}
exposes the \ac{TLS} handshake to new vulnerabilities,
especially in the case of \ac{SECH} 3 where the \var{enc}
value is distinguishable from a uniform random value.
This aspect of the designs requires more thorough cryptographic analysis.

Another appealing aspect of \ac{SECH} as opposed to \ac{ECH}
is how it could provide deniability.
If a user's history of outer \ac{SNI} values has been recorded,
this is not necessarily evidence that the user actually visited
those sites, since the user may have been using \ac{SECH}.
This holds true even if \ac{SECH} has never been deployed since
the user can claim to have used/deployed \ac{SECH},
and the attacker cannot determine whether they are telling the truth.

Still, we find it unlikely that our \ac{SECH} designs will gain
widespread adoption solely for the privacy benefit offered,
since the \ac{ECH} draft is already gaining adoption and has
been much more thoroughly analysed.
However, the censorship-circumvention use case of \ac{SECH} may have greater appeal.

\section{Future Work}

Our design of \ac{SECH} 2 specifies only \ac{AES}-128-\ac{GCM} as the \ac{AEAD} cipher for servername encryption, since
it is specified that a \ac{TLS}-compliant
server must implement this cipher.
Since the ciphertext is sent in the first flight before any negotiation takes place this simplifies the design.
The design could be extended to allow other ciphers,
as well as connection-specific negotiation of the servername encryption cipher.
For example, the \ac{AEAD} cipher could be negotiated as the preferred cipher in the 
\var{cipher\_suites} list.

Our design of \ac{SECH} 2 uses a 24 byte inner random and 12 byte inner payload.
There are several avenues we could explore in order to expand the available bandwidth for the inner payload (the \ac{SNI} and \ac{ALPN}).
For one, rather than transmitting the 24 byte inner random the inner random could be derived from the shared secret.
For split-mode, this derived inner random would need to be embedded in the \ac{CHI}.
Since the \ac{AEAD} \nonce is not needed by the backend server we could put part of the inner random in the first 12 bytes of the \ac{CHI}'s \var{random}.
More of the inner random could go in the \ac{SNI} extension data field, rather than transmitting a string of 0s, and finally,
the backend server does not need the \ac{AEAD} \ac{MAC}, so these 16 bytes could be used for the inner random.
With these changes we can increase the size of both the inner random {\em and} the \ac{SECH} payload.

For our design of \ac{SECH} 2 we abandoned the goal of offering
split mode in order to include a mitigation
against a \ac{HRR} hijacking attack that could break \ac{PC}.
We have not ruled out the possibility of both mitigating the
\ac{HRR} hijacking attack {\em and} offering split-mode.
This problem is a direction for future research.

More ambitiously, however, it would be ideal to find
a mitigation against the \ac{HRR} hijacking problem that
does not involve excessive cover messages and bandwidth.

Our \ac{SECH} 2 design allows bootstrapping access using the \ac{PSK} system.
This means that any client can access
the \ac{SECH} 2 server without the need for a
pre-established secret.
This entails that channel-level \ac{PC} can be broken simply by trying to connect to the server using the \ac{PSK} bootstrap method.
An alternative design could protect the channel-level \ac{PC} by ensuring that only \ac{PSK} established by the \varsechlongtermkey{}
could be used for connecting to the \ac{SECH} 2 backend.
This could be implemented in application code
without additional changes to the protocol,
since the specific format and semantics of \ac{PSK} tickets is not part of the \ac{TLS} 1.3 spec,
and this flexibility could be leveraged to
assure channel-level \ac{PC}.

\subsection{Non-uniform \var{enc}}

The non-uniformity of the \ac{HPKE} \var{enc} leaks
information that \ac{SECH} 3 is being attempted.
One approach to mitigate this would be shape the distribution
of a sequence of \var{enc} values by rejecting some of them.
For example, say the first bit has probability $2/3$ of being 1.
If we reject half of the \var{enc} values for which the first bit is 1
then the distribution of accepted \var{enc}s' first bit will be uniform.
Quantifying the true distributions for \ac{HPKE} \ac{KEM}s,
in particular \ac{X25519},
and constructing rejection algorithms to shape the distribution of \var{enc},
could be used to improve the stealth of \ac{SECH} 3.

\subsection{Inner Servername Compression}
Mixing compression and encryption can expose a system
to new attacks against confidentiality.
For instance the \ac{CRIME} and \ac{BREACH} attacks
have been successfully mounted against various versions of \ac{TLS}.

To avoid these types of attacks we have neglected
to pursue a design incorporating compression of the inner servername.
However, considering the severe limitation of a maximum of 12 bytes
to encode the inner servername in \ac{SECH} 2 and \ac{SECH} 3,
there is a strong temptation to incorporate servername compression
in order to increase the maximum inner servername length.

One approach would be to use a more efficient encoding leveraging
the restrictions on domain names specified in RFC 1034 \citep{rfc1034}.
For instance the character set for domain names is much smaller than the 255 possible ASCII values, meaning each character could be encoded with fewer than 8 bits.

Another approach would be to send the hash of the servername,
where the hash can be an arbitrary length.
This would entail an engineering burden to deal with the case
of hash collisions, but this can be easily overcome.

\subsection{SECH 1: Secretless Stealthy Encoding}
\label{sec:sech1}
There are potential use-cases
for an \ac{SECH} variant
that uses no encryption (and thus
needs no secret)
but such a variant should more properly be called just `stealthy SNI' or `stealthy CH'.

An advantage of such a variant is the very low coordination/infrastructure requirements to get this working.
Client and server simply need to run the same
protocol, with no need for out-of-band secret sharing,
or public key distribution.

Our design for \ac{SECH} 1 does {\em not} offer confidentiality of the \ac{SNI} or \ac{ALPN}.
The aim here is purely to circumvent censorship
in the case of a naïve censor.

As a censorship circumvention method this can be easily detected and prevented,
but this is still possibly useful transiently and at a small scale.
If only a small number of Internet users are using the scheme
then it might successfully evade censorship for a time,
before the censor invests the resources to block it.

% This approach could be one element in a strategy of circumvention in depth: using lots of different circumvention methods (including this insecure one) in order to increase the cost of censorship for the censor.
