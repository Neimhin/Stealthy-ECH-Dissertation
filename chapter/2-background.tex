\chapter{Background}

\section{TLS}

% [x] TLS the most popular way to protect internet connections
\ac{TLS}, being the underlying confidentiality/authentication protocol for \ac{HTTPS},
is probably the most widely used protocol for securing Internet traffic today.
The specifications for \ac{TLS} have been developed and published by the \ac{IETF},
which is an open community dedicated to developing technologies to make the Internet better
for Internet users.

% [ ] TODO: diagram of the client/server and the insecure channel

Apart from confidentiality and authentication one of the central
design goals for \ac{TLS} has been scalability.
An important aspect of \ac{TLS}'s scalability is the ability
for \ac{TLS} servers to operate statelessly.
And another aspect of the scalability story is how
authentication is achieved using a chain-of-trust model.
These aspects are arguably secondary to the confidentiality and authentication
requirements of \ac{TLS} but should still need to be taken into consideration
when designing extensions to \ac{TLS}.

% \subsection{Early Years of SSL}
% [ ] the \ac{SSL}  protocol was, in essence, developed to facilitate credit/debit card transactions (and hence e-commerce/internet shops) on the internet
\ac{TLS} is a successor to the \ac{SSL} developed
by Netscape Communications.
One of the essential motivations for the development of \ac{SSL}
was to enable secure electronic commerce on the web.
Message confidentiality is the property that allows a user
to send their payment card details to exactly one recipient and no-one else.
Message authenticity gives the user a gaurantee that the message
was sent to/from the appropriate party.
And message integrity ensures that a transaction of say €20 isn't
maliciously or accidentally modified into a transaction of €2000.

Shortly after the release of Netscape's first web browser, Mosaic,
Netscape finished the design of \ac{SSL} 1.0 in 1994,
but this design was never shipped in a commercial product
due to the discovery of various security flaws.
In late 1994 Netscape released Netscape Navigator which
came equipped with \ac{SSL} 2.0.
Microsoft joined the browser battles in 1995 with the release
of Internet Explorer, and developed an \ac{SSL} alternative
called \ac{PCT}.
In late 1995 \ac{SSL} 3.0 was released, which was the last
major version update to under the moniker \ac{SSL}.

In 1996 the \ac{IETF} formed the \ac{TLSWG} with
the responsibility to standardize a transport layer security
protocol.
The basis for that work was \ac{SSL} 3.0, but
under the direction of the \ac{IETF} it was decided to rebrand as \ac{TLS}.
\ac{TLSWG} has since overseen the standardisation of
\ac{TLS} and \ac{DTLS} versions 1.0, 1.1, 1.2, and 1.3.

\ac{TLS} 1.3 is a major update over \ac{TLS} 1.2, despite
the name indicating only a minor update.
\ac{TLS} 1.3 introduces new functionality such as 0\ac{RTT} data,
reduces the number of round-trips needed to perform the handshake (in most cases),
demands \ac{AEAD}-type symmetric ciphers in all cipher suites,
and deprecates the use of static \ac{DH} in favor of \ac{DHE},
among other changes.
This is to say \ac{TLS} 1.2 and 1.3 are significantly different protocols.
In this dissertation we aim to design extensions to \ac{TLS} 1.3.
Our extensions will not support earlier \ac{TLS} versions,
however, it is often the case that
a \ac{TLS} server supports multiple versions simultaneously.
To facilitate this securely \ac{TLS} has measures
to prevent downgrade attacks, i.e. attacks where
a client and server end up negotiating a lower protocol
version than the maximum mutually supported version.
Our protocols need to be designed in a way that does
not interfere with protections against downgrade attacks.

% [ ] to achieve this we need the credit card details to be confidential, but the client (who owns the credit card) also needs to know they are sending their details to a legitimate server (authentication) 

% [ ] it turns out that application-data-confidentiality and server-authentication are much more generically useful and valuable for the security and privacy of netizens, so \ac{SSL}  was standardized by the IETF under the new name TLS

% \subsection{Standardising TLS}

% [x] OSI transport layer vs TCP/IP transport layer
The term `transport layer' in \ac{TLS} is a cognate of the \ac{OSI} model's 'transport layer', which is a conceptual division of the roles/responsibilities of different pieces of software, and which helps in identifying appropriate levels of abstraction for programming models/interfaces.
An alternative to the \ac{OSI}'s seven layer model is the simpler 4 layer \ac{TCP}/\ac{IP} model consisting
of the network access, internet, transport, and application layers.
\ac{TLS} provides functionality for transport control (message ordering, connection maintenance),
as well as session maintenance, data compression and encryption,
which in terms of the \ac{OSI} model corresponds to all three of the transport, session, and presentation layers.
Ideally, a protocol that runs at the transport layer should leverage
the contracts offered by services of the superior layer,
and consume the services of the inferior layer,
while treating data from those layers as opaque. % TODO: a bit dodgy

% [x] TCP and UDP
Two important protocols in the transport layer are \ac{TCP} and \ac{UDP}, and \ac{TLS} (or \ac{DTLS} respectively) wraps the TCP and UDP protocols providing confidential/authenticated versions of these protocols. IP packets are not large enough to transfer arbitrary messages in a single packet, so messages are broken up into sequences of packets, and the transport layer protocols are concerned with how the larger messages are split up into IP packets, as well as trade-offs between reliability, latency, and network congestion (TCP prioritises complete delivery of the message in the proper order, whereas UDP prioritises low latency).

% [x] TLS API similar to socket API
% The \ac{API} for establishing a \ac{TLS} connection is designed to be as similar as possible to that of establishing a \ac{TCP} connection, or at least to not be gratuitously different. Some differences are that there are additional interfaces for configuring authentication and new kinds of failures that can occur with \ac{TLS} connection which are not possible for a \ac{TCP}.

% [x] pure separation of OSI model layers
If we were to adhere strictly to the \ac{OSI} model then all information pertaining to other layers of the \ac{OSI} model (in particular for our purposes application, presentation, and session) would be opaque from the perspective of (and encrypted under) TLS (i.e. we would have a pure separation of the \ac{OSI} layers), but over the years \ac{TLS} has incorporated various application-level information in extensions (\ac{SNI}, \ac{ALPN}) which are potentially used to reverse proxy routing (i.e. which
backend server should terminate the \ac{TLS} connection) and are thus not encrypted.

\subsection{TLS 1.3}

\ac{RFC} 8446 (\citep{rfc8446}) which specifies the current recommended version of \ac{TLS} (1.3)
was published in 2018 after about 4 years
of active development.
The first draft of \ac{TLS} 1.3 was published
in October 2014, a bit over a year after
Edward Snowden released classified documents
detailing the \ac{NSA}'s surveillance activities.
The Snowden revelations imbued the \ac{IETF}'s
\ac{TLS} development efforts with a renewed vigor.

% [x] development of TLS 1.3 largely motivated by Snowdonia

% [ ] TLS 1.3 was a major update to the protocol compared to TLS 1.2 (although the significance of the update is not reflected in the name)

Earlier drafts of \ac{TLS} 1.3 were found to have significant deployment failure/issues due to middleboxes incorrectly implementing earlier versions.
The \ac{TLS} protocols were designed with
extensibility in mind, such that for instance,
it should be possible to update the version
number in \ac{TLS} messages without causing breakages.
It was found that middleboxes failed to correctly
ignore higher version numbers,
resulting in the somewhat unfortunate kludge
in \ac{TLS} 1.3 where the version number field of the
\ac{CH} message
has the value associated with \ac{TLS} 1.2,
and the true version number is indicated
in a message extension field.

Another kludge is that the \ac{HRR}
message type has the same message type
identifier as the \ac{SH} message.
To distinguish between the \ac{HRR} and \ac{SH} message types we actually
have to look at a magic value in the
\var{random} field.
The presence of that magic
value indicates the message is of type \ac{HRR}.

% [x] various unfortunate cludges such as the version field indicating 1.2 and true version in \var{supported\_versions} extension, but these are not a big problem

% [x] due to its' importance on the internet and hence for the global economy TLS 1.3 has seen considerable cryptographic and security analysis, which endows it with lots of confidence from systems architects, and so TLS (and especially TLS 1.3) is being incorporated into many systems beyond just client/server web connections (e.g. email, DNS, connections between servers/databases in data centers etc.

The massive adoption of \ac{TLS} 1.3 on the web,
for email,
as a component in Zero Trust network architectures,
and in other environments has encouraged
significant effort by cryptanalysts
and hackers to establish, by
manual, mechanistic and formal means,
the security properties of the protocol.
It is not necessarily any one formal analysis or proof
of \ac{TLS} 1.3's security that should give us confidence in its
security, but rather the fact that the combined effort of
many independent researchers has failed to reveal
any fatal flaws.

% [x] a common flow on the web is for a browser session to establish many concurrent connections, and some features of TLS 1.3 facilitate performing these many connections scalably, 0-RTT (careful!), session tickets/resumption, stateless tickets
\ac{TLS} was designed for the web, and some of its
features are specifically intended to minimize
the security overhead experienced by web users.
For instance, a common flow on the web is for the web browser
to first load a small \ac{HTTP} resource such as a \ac{HTML}
file, which in turn references lots of other resources which need
to be loaded to render the page properly.
When fetching these resources over \ac{TLS} this means it is very
common to establish an initial connection to the server
quickly followed by 10s or hundreds of connections to the same server.
The \ac{TLS} 1.3 \ac{PSK} mechanism helps to reduce
the amount of bandwidth needed for each of the subsequent connections.
It turns out that the \ac{PSK} mechanism will be useful in the
design of our \ac{SECH} protocols.

\subsection{CAs and the Web PKI}
\ac{TLS} 1.3 offers two mechanisms to achieve authentication;
certificate based authentication and the external \ac{PSK} authentication.
Certificate based authentication is the standard way
to authenticate servers on the web.
This certificate based scheme involves at least three parties; the client
(requesting authentication), the server (proving authenticity),
and a trusted third party, the \ac{CA}.
Usually, client applications are configured (e.g. by the \ac{OS} or browser
manufacturer) with a list of trusted root \ac{CA}s.
Server operators register with the root \ac{CA}
(either directly or more likely via a subordinate \ac{CA})
to acquire a certificate
signed using \ac{PKC} by the \ac{CA}.
This certificate is essentially an attestation that
the servername, e.g. \var{example.com}, is owned/controlled by the server.
The certificate contains a servername, a public key
and the \ac{CA}'s cryptographic signature.
The cryptographic signature ensures that if the server
were to tamper with its signature, e.g. by changing the servername or the public
key,
then a client would be able to detect this tampering.
The private key associated with the certificate's public key is controlled by the server.

In \ac{TLS} the server authenticates itself by providing the certificate
as well as a cryptographic signature (in the \ac{CV} message) using the private key associated with the certificate. The signature digests the transcript of the \ac{TLS} handshake messages, which means the signature serves as proof that the server currently
possesses the corresponding private key,
and that the \ac{CV} message has not been replayed from some other session.

The mechanism for gauranteeing freshness of the server's \ac{CV} message is related to the \var{random} fields in the \ac{CH} and \ac{SH} messages.
This field must (in compliance with \cite{rfc8446}) consist of 32 octets generated
uniformly at random,
either using a true source of randomness or a secure \ac{PRNG}.
For \ac{SECH} we are proposing that these random values
can be replaced (in full or in part) with non-random
encrypted values or cryptographically hashed values.
A potential danger of our approach is that it could weaken
or destroy \ac{TLS} protections against replay.

The Web \ac{PKI} is a constellation of actors
and organisations that facilitates
servers acquiring certificates
which clients can trust.
In the past it was common for \ac{CA}s to charge a fee to servers,
but in recent years the Let's Encrypt \ac{CA}/service,
which offers free and automated certificate acquisition and renewal
has been influential in more rapid adoption of \ac{TLS} by
web server operators.
The Let's Encrypt service makes use of the \ac{ACME} protocol,
which establishes ownership of a servername by
asking the server to (automatically) complete a series of challenges.

An implication of this is that we can usually interpret
server authentication to mean roughly
``the given servername resolves to a set of \ac{IP} addresses in the \ac{DNS},
and the \ac{CA} has at some point in the past gathered
evidence that those \ac{IP} addresses are serviced by
servers which possess the relevant private key''.

There have been successful attacks against \ac{TLS}'s certificate-based
authentication system by targeting and compromising \ac{CA}s.
The \ac{CA} company DigiNotar was discovered to have been hacked in 2011, and it was found
that fake certificates had been issued which enabled a large \ac{MITM} attack affecting about 300,000 Internet users in Iran.
Domains affected by the attack included \var{google.com}, \var{update.microsoft.com}, and \var{www.cia.gov} \citep{van2013diginotar}.

Something to note about certificate based authentication, especially
in web browsers, is that the typical user likely never audits the list of
trusted root \ac{CA}s,
so implicitly the user places their trust in the
client application manufacturer,
which in turn decides which \ac{CA}s to trust.

% [x] CAs and the Web PKI are all about server authentication, linking server names to trusted servers

\subsection{SNI}
The \ac{SNI} extension to \ac{TLS} may be included in the \ac{CH} message, and it allows the server to determine which virtual server the client wants to connect to in the case that there are multiple virtual servers sharing a single public network address, i.e. the domains are co-located.

% [ ] TODO: diagram of co-location

As \cite{hoang-2020-assessing-privacy-benefits-sni-encryption} have noted, however, the domain name reveals semantic information about the internet user. For example an on-path network observer could gain information
based on the \ac{SNI}
about a users health (\var{hiv.gov}, \var{cancer.gov}), religion (\var{islamicity.org}, \var{quran.com}), gender identity (\var{lgbt.foundation}, \var{gaycenter.org}), and more (examples after \cite{hoang-2020-assessing-privacy-benefits-sni-encryption}).

When the \ac{SNI} extension was first defined and gaining adoption it was reasonable not to consider it as an important privacy leak.
For one, the \ac{SNI} at the time could usually be accurately inferred from the network address.
It is necessary to send the network address in the clear in order for the \ac{IP} to work.
These factors meant that securing confidentiality of the \ac{SNI} would have done almost nothing for the privacy of the user.
Recently though, \cite{hoang-2020-assessing-privacy-benefits-sni-encryption} studied the relationship between IPv4 address
and server/domain names on the Internet and found that for many websites
(around 20\%) there is still a one-to-one mapping from IPv4 to domain name,
which an on-path observer can infer.
For these websites encrypting the \ac{SNI} would provide no privacy benefit to users.

Domain co-location, the practice of hosting large numbers of virtual servers at a single network address (i.e. an IPv4 or IPv6 address), has become increasingly common with the rise in popularity of large \acp{CDN} and cloud hosting providers. One of the reasons for this practice is the scarcity of IPv4 addresses making these addresses a valuable and costly resource \citep{mueller2008scarcity}. Having many domains share an IPv4 address has a significant impact on assuaging the out-sized demand for these addresses.

% [x] when first introduced it was thought that the SNI was not a privacy leak because the SNI could be fairly accurately inferred from the IP

Nowadays, it is common for a domain to be co-located with hundreds or thousands of other domains.
This means that if the \ac{SNI} extension could be hidden then an observer would only attain a small amount of information about the true \ac{SNI} by inference from the \ac{IP} address.
\cite{esni} call the set of co-located domains the anonymity set.
A larger anonymity set implies greater privacy (assuming the \ac{SNI} remains confidential).

The benefit to privacy of domain name encryption is related to the size of the domain's anonymity set which we'll call $k$. For a domain with $k$ co-located domains an on-path observer has a $1/k$ probability of guessing the domain name from the \ac{IP} address,
assuming no {\em a priori} knowledge about the probability of each servername in the anonymit set.
However, \cite{hoang-2020-assessing-privacy-benefits-sni-encryption} discovered that more popular domain names tend to have smaller anonymity sets (smaller $k$), while less popular domain names are more likely to have larger $k$, meaning there is a greater potential for user-privacy for these less popular websites.
One of \citeauthor{hoang-2020-assessing-privacy-benefits-sni-encryption}'s \citeyear{hoang-2020-assessing-privacy-benefits-sni-encryption} recommendations is that domain hosting providers should group more virtual servers behind single \ac{IP} addresses in order to increase the value of $k$ for each domain and thus increase the potential privacy-benefit of servername encryption.

In actual fact, an individual website is often available via multiple different \ac{IP} addresses. A \ac{CDN} will make a domain available from different \ac{IP} addresses depending on the geographic location of the client in order to minimize latency. In such a case each \ac{IP} address may host a different set of domain names, meaning the anonymity set is different depending on the geographic location of the client. \cite{hoang-2020-assessing-privacy-benefits-sni-encryption} simplify their analysis by computing $k$ as the median anonymity set size over all \ac{IP} addresses serving the given domain.

\subsection{ALPN}
Another privacy-leaking  extension to \ac{TLS} is the \ac{ALPN}. The \ac{ALPN} extension can be
included in the \ac{CH} and lets the server know the client's preferences
of application-layer protocol to use once the handshake has completed.
For instance, a client might prefer to use \ac{HTTP}/2, but also supports \ac{HTTP} 1.1.
The \ac{ALPN} extension value for this case is presented in Listing~\ref{lst:alpn-example-h2}.

\begin{listing}
\begin{Verbatim}[frame=single]
Handshake Type: Client Hello (1)
...
Extension: application_layer_protocol_negotiation (len=14)
    Type: application_layer_protocol_negotiation (16)
    Length: 14
    ALPN Extension Length: 12
    ALPN Protocol
        ALPN string length: 2
        ALPN Next Protocol: h2
        ALPN string length: 8
        ALPN Next Protocol: http/1.1


Handshake Type: Server Hello (2)
...
Extension: application_layer_protocol_negotiation (len=5)
    Type: application_layer_protocol_negotiation (16)
    Length: 5
    ALPN Extension Length: 3
    ALPN Protocol
        ALPN string length: 2
        ALPN Next Protocol: h2
\end{Verbatim}
\captionsetup{width=0.8\linewidth}
\caption[Example ALPN Value]{\label{lst:alpn-example-h2}Example \ac{ALPN} value where the client prefers \ac{HTTP}/2 but also supports \ac{HTTP}/1.1}
\end{listing}

The \ac{ALPN} serves two purposes: 1. identity selection, and 2. fewer round-trips.
The \ac{ALPN} allows the server to select an appropriate identity based on the application
protocol negotiated before the handshake completes, e.g. if for some reason a server were listening for both \ac{SMTP} and  \ac{HTTP} connections on the same port
the server could serve a different authentication certificate for each protocol.
Secondly, without the \ac{ALPN} the client and server would have to perform protocol negotiation after the handshake had finished
which might entail an additional flight of messages or a full additional round-trip.
The \ac{ALPN} extension allows
this protocol negotiation step to piggy back on the \ac{TLS} handshake, thus cutting out a round-trip.

The \ac{ALPN} is sent in cleartext (both from the client and the server), so on-path network observers
can easily determine which application protocol is being used for a \ac{TLS} connection.
If the application-protocol can be inferred then it may also be possible to determine the underlying server-identity,
which is a further privacy leak.
The \ac{ALPN} is both a privacy-leak and a potential discriminator for censorship.
For these reasons \cite{rfc8744-issues} have noted \ac{ALPN} encryption as an additional design goal
for any servername encryption scheme.

Something to note, though, is that the reduction of round-trips is a {\em raison d'être} of the \ac{ALPN}.
If an \ac{ALPN} encryption scheme involved an additional round-trip it would defeat the original purpose of the \ac{ALPN} extension.
This is one of the reasons we focus our attention in this dissertation on domain-name and \ac{ALPN} encryption schemes that
do not increase the number of round-trips compared to regular \ac{TLS}.

\subsection{Fixing the SNI leak: ESNI}

% [x] How the SNI is used in censorship (DPI) and pervasive monitoring.

\cite{chai2019importance} were able to detect \ac{SNI} filtering by the \ac{GFW} by leveraging the fact that the \ac{GFW}'s censorship is bidirectional (traffic whose source is outside the \ac{GFW}, and whose destination is inside, is filtered. Similarly vice versa).
\cite{chai2019importance} observed domain censorship by way of injected \acp{RST} packets.
The censorship was triggered by the presence of block-listed servernames in the \var{server\_name} extension of the \ac{TLS} \ac{CH}.
It was found that 21,446 sites (of 1,000,000 tested) suffered \ac{SNI} filtering,
but that \ac{SNI} filtering was in almost all cases used in conjunction with another censorship mechanism
(such as \ac{IP} blocking or \ac{DNS} hijacking).
A residual censorship period of 60 seconds was also observed on the censored source IP, destination IP, and destination port triple, meaning all traffic was blocked between the offending client and server for 60 seconds after an \ac{SNI}-based blocking.
\ac{SNI} filtering is an example of \ac{DPI},
which is a somewhat sophisticated technology that implies the development of at least
some custom software components
(if not also custom hardware)
by the censor to perform the \ac{SNI} filtering.
It has been a general trend that the Chinese
government has been willing to make
significant investments into the development
of the `Golden Shield Project',
the Internet surveillance and censorship project
known to most as the \ac{GFW} \citep{chandel-2019-great-firewall}.

% [ ] How does ESNI work? What are its properties?

% [x] How widely did ESNI get deployed? https://ieeexplore.ieee.org/abstract/document/9780903

\cite{guan2021largescalemeasurementofesni} measured \ac{ESNI} deployment from June to August 2020 and found that the protocol was gaining significant adoption, with 0.73\% of global sites and 11.86\% of the `Alexa~Top~1~Million' sites offering \ac{ESNI}. \cite{guan2021largescalemeasurementofesni} assess that \ac{ESNI} adoption was primarily driven by Internet companies, Cloudflare and Mozilla in particular, which is in indication of market interest in the added privacy.

One of the downsides of a stealthy variant of \ac{ECH} will be that this kind of
empirical measurement of employment and adoption will not be possible.
By design we should never know exactly how widespread \ac{SECH} adoption is.

% TODO REPETITION OVER-BLOCKING
The \ac{ESNI} extension to \ac{TLS} (which has since evolved into \ac{ECH}) uses public key cryptography to encrypt the server name in a new extension called the \var{encrypted\_server\_name} which was distinguised by the \var{0xffce} \var{ExtensionType} identifier.
This extension type identifier is sent in the clear (in the \var{ClientHello}), and thus it is trivial for a censor to block all \ac{ESNI} traffic.
However, when the servername is protected with \ac{ESNI} it is not possible for the censor to surgically block just one domain name on a given channel, the censor is forced to decide whether to block all servernames on that channel, or none of them.
\ac{ESNI} was designed with the hope that the economic cost of over-blocking sites
(by blocking all \ac{ESNI}-based connections)
would be a strong enough incentive to cause censors not to block any \ac{ESNI} traffic.
In a global measurement experiment \cite{chai2019importance} did not report any ESNI blocking,
but shortly thereafter,
on the 20th of July 2020 there were reports of \ac{ESNI} blocking by the \ac{GFW}.
\cite{bock2020censorship} confirmed this censorship was occurring based on the \var{0xffce} identifier in the \var{encrypted\_server\_name} extension.
\cite{bock2020censorship} found that by changing the \ac{ESNI} extension ID from,
e.g., \var{0xffce} to \var{0x7777}, \ac{ESNI} would no longer trigger blocking by the \ac{GFW}.
They found that omission of the \ac{SNI} did {\em not} trigger blocking by the \ac{GFW}.
However, changing the \ac{ESNI} extension ID is not a sustainable or long-term circumvention technique,
it merely allowed \cite{bock2020censorship} to ascertain
the method the \ac{GFW} was using to detect and then censor \ac{ESNI}.

% [ ] Over-blocking as a deterent for deploying ESNI (empirical example in China)
% One of the hesitations 

% [ ] Implications of regional blocking

\subsection{ECH}

% [ ] ESNI being blocked, and actually more parts of the ClientHello that are sensitive, so let's encrypt the whole thing!
The \ac{ESNI} proposal has since evolved into
\ac{ECH} which seems to be close to
finalisation, and will likely soon
be published as an \ac{RFC}.


% [x] Current status of ECH development and deployment: IETF draft, option on cloud
For a time Cloudflare allowed customers to enable \ac{ECH},
but this option was disabled in October 2023
for `unrelated reasons',
and there is still no confirmed timeline for
\ac{ECH} to be reenabled \citep{cloudflare-ech-disabled}.

In \ac{ECH} more parts of the \ac{CH} message
can be encrypted compared to \ac{ESNI},
including \ac{TLS} extensions,
which offers flexibility for extending
\ac{TLS}\ac{ECH} in future with privacy-sensitive data
in the first flight of client messages.

The design goal ``Don't stick out'' is addressed
in \ac{ECH} using a technique called \ac{GREASE}.
The \ac{GREASE} technique was first put forward as
a method of mitigating network ossification due to
misbehaving protocol implementations.
It involves replacing variable \ac{TLS} message field values with random values to discourage
implementations relying on the presence or absence
of the values to function.
A \ac{GREASE}'d \ac{ECH} handshake is one where the \varech{} extension is present but with the \var{config\_id} and the \var{payload} field, which usually contains the encrypted \ac{ECHI}, replaced with random values.
From the perspective of a passive observer \ac{GREASE}'d \ac{ECH} is designed to be
indistinguishable from regular \ac{ECH}.
However, the first purpose mentioned by \cite{esni} for the \ac{GREASE}'d \ac{ECH} is actually
to enable the client to request an up-to-date \var{ECHConfig} value
from the server,
allowing the client to perform
a real \ac{ECH} request subsequently.
Also, the \ac{ECH} draft does not aim to achieve indistinguishability of \ac{ECH} and \ac{GREASE}'d \ac{ECH} when the adversary
is an active attacker.

% [x] Particular design goal: don't stick out, implemented with GREASE

% [x] Could stick out even less if an ECH handshake looked exactly like a regular TLS 1.3 handshake. The fact that ECH sticks out (/requires GREASE not to stick out), makes it technically easy for ECH to be blocked entirely, especially by state-level censors (China, Iran, South Korea).

% [ ] IETF pursues ECH for relatively pure reasons, privacy enhancement, less censorship. Why are companies like cloudflare and google pursuing ECH if there's a risk of it being blocked? -> google.com currently blocked in china, could google get access to a 4 billion person market by deploying ECH?
The ostensible motivation of the \ac{IETF} in
developing \ac{ECH} is for the benefit of individual Internet users, not \acp{ISP},
corporations, \acp{CDN}, etc. as implied by \cite{rfc8890internet-is-for-end-users}.
For commercial entities like Cloudflare the
pursuit of enhanced user-privacy might be part of
a branding and marketing strategy.
But another motivation could be the potential of
hugely expanding the addressable market by
circumventing or undermining large scale
censorship campaigns such at the \ac{GFW}.
For example, \var{google.com} is censored by the \ac{GFW}.
Could Google expand its consumer base to
China's population by getting behind the roll out of \ac{ECH}?

% [ ] What would it mean if ECH were made a default in popular clients like Google Chrome, Firefox?

% [ ] Google and China play Chicken (Comic illustration?)
\section{Internet Censorship}

RFC 9505 \citep{rfc9505} names three actions taken by censors: prescription, identification, and interference.

Prescription is the process of determining what to censor, and there is little that can
be done on the technical front to influence the prescription process,
but identification and interference can be circumvented by technical means.
Identification is the process of classifying specific traffic,
for instance when \ac{ESNI} is
classified based on the presence of its \var{ExtensionType} identifier.
Interference is when a censor participates in
active measures to curtail previously identified traffic.

In this work we are concerned with designing
an identification-resistant \ac{SECH} protocol in the hopes that 
preventing identification will mitigate interference.

Censorship is usually facilitated by a centralisation of control of some aspect of the Internet, e.g. the name servers, Internet Exchange Points, CAs, but censorship can be (and is) also implemented by service providers such as in Google search results, social media (e.g. Donald Trump's Twitter ban) etc.

Censorship is easier and cheaper when data are not encrypted. When data are encrypted a censor might decide to over-block,
i.e. block all communications on a given channel
in order to achieve the blocking of
particular material believed to be
carried on that channel.
The greater a censor's abitility to identify traffic accurately,
the more granular can be the interference.
Limiting a censor's ability to identify traffic
causes a censor to require a greater degree of over-blocking in order
to achieve its prescribed goals.

Residual censoring is when a censor blocks traffic between two endpoints as a sort of punishment after identifying communications between the two endpoints.
Non-technical punishments for attempts to circumvent censorship, such as social credit reductions, imprisonment, confiscation can also be employed.

Port blocking can be used to target specific protocols, such as \ac{HTTPS}, possibly forcing netizens to fall back to \ac{HTTP}.
Using \ac{HTTP} instead of \ac{HTTPS} facilitates more fine grained and less expensive identification and interference.

% TODO: repetition
In this work we are considering the scenario where \ac{TLS} 1.3 is not generically blocked,
but where domains are possibly blocked based on the \ac{SNI}.

% [ ] Note that censors often employ a wide range of technical measures to block the same thing (e.g. IP/TCP header-based blocking and DPI and residual consoring and DNS poisoning), meaning that circumventing one of those measures might not be enough to gain access to the blocked comms channel.

% [ ] \subsection{Active Probing}

% [ ] what is active probing https://kevinbock.phd/pdf/2023-usenix-fully-encrypted-traffic.pdf

% [ ] active probing against SSH tunnels, Shadowsocks tunnels \cite{wu2023great}

\section{Closely Related Work}

Issues and requirements for the encryption of the \ac{SNI} and other fields in \ac{TLS} 1.3 were laid out by \cite{rfc8744-issues}, and the main effort by the \ac{IETF} community to design a solution has been focused on the \ac{ECH} draft.
However, various other approaches
have been proposed and discussed,
including the use of alternative tunnels such as those using
TOR, \acp{VPN}, or shadowsocks.

\cite{rfc8744-issues} discuss briefly the idea of tunnelling \ac{TLS} or \ac{HTTPS} in another \ac{TLS} connection. The idea is that the client first establishes a \ac{TLS} connection to the client-facing server. Then the client conducts a second \ac{TLS} handshake, but with handshake messages (including the \var{ClientHello}) being protected with the application traffic secrets of the first handshake.

One of \citeauthor{rfc8744-issues}'s arguments against this design is that it involves double encryption of every packet. When the client sends application data it is first encrypted with the backend server's key, and then again with the client-facing server's key. The client-facing server unwraps the first layer of encryption, and the backend server unwraps the second recovering the plaintext. The second argument against \ac{HoT} is the requirement for the client-facing server to decrypt all relayed application data. These aspects are considered too onerous, i.e. it would cause excessive latency for the end-user, it would roughly double the expense of operating a client-facing server, and it would increase the server's vulnerability to \ac{DoS}. The issue of increasing the expense for the client-facing server is important, because there is only weak economic incentive for large \acp{CDN} and service providers to offer \ac{SNI} encryption, so a significant increase in operational cost caused by \ac{SNI} encryption could destroy its possibility of widespread deployment.

One of the good properties of \ac{HoT} is that the client authenticates both the client-facing server and the backend server, which mitigates \ac{MITM} attacks. This would also allow the client to apply more granular policies of how to behave depending on the trust level endowed to the client-facing server.

One of the key challenges I see with \ac{HoT} is that the second (inner) handshake may be detectable via traffic analysis. The sequence of messages of the inner handshake, although encrypted, will by default have an identifiable fingerprint due to the consistent pattern of messages: \var{ClientHello}, \var{ServerHello}..\var{Finished}, \var{Finished}. This tick-tock-tick pattern of messages might stand out quite a bit on the Internet, since it is very common for a Web request to consist of just tick-tock, e.g. a \ac{HTTP} request and response.
Obfuscating the existence of the inner handshake might require ongoing adaptation to other Internet traffic, which is a serious challenge to attaining a completely stealthy deployment.

The focus of this work is to purse an \ac{SNI} encryption scheme that does
require additional parties other than the client, client-facing server, and backend server,
and also which does not entail additional handshake round-trips compared to regular \ac{TLS}.
For these reasons we do not discuss the alternative tunnel approaches further.
