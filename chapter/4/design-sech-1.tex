\section{SECH 1: Secretless Stealthy Encoding}
\subsection{Motivations and Deployment Scenarios}

There are potential use-cases
for an \ac{SECH} variant
that uses no encryption (and thus
needs no secret)
but such a variant should more properly be called just `stealthy SNI' or `stealthy CH'.

An advantage of such a variant is the very low coordination/infrastructure requirements to get this working.
Client and server simply need to run the same
protocol, with no need for out-of-band secret sharing,
or public key distribution.

Our design for \ac{SECH} 1 does {\em not} offer confidentiality of the \ac{SNI} or \ac{ALPN}.
The aim here is purely to circumvent censorship
in the case of a naïve censor.

As a censorship circumvention method this can be easily detected and prevented, but this it is still possibly useful transiently and at a small scale.
If only a small number of Internet users are using the scheme
then it might successfully evade censorship for a time,
before the censor invests the resources to block it.

This approach could be one element in a strategy of circumvention in depth: using lots of different circumvention methods (including this insecure one) in order to increase the cost of censorship for the censor.

\subsection{Design}
It turns out that \ac{SECH} 1 can be implemented 
by running \ac{SECH} 2 (which we'll describe below),
except using a publicly known \varsechlongtermkey{}.

The basic idea is to use a symmetric key encryption to encrypt the
inner \ac{SNI} and \ac{ALPN}, and then replace the \var{random}
and \varlegacysessionid{} fields of the \var{ClientHello}
with the ciphertext. Since the ciphertext is pseudorandom
a censor is forced to perform a decryption operation using
the publicly known key in order to detect \ac{SECH} 1.

Some of the design choices we lay out for \ac{SECH} 2 are overkill
for \ac{SECH} 1.
For instance the acceptance confirmation signal
could be dropped in the case of \ac{SECH} 1.
Due to the limited use-cases for \ac{SECH} 1, however,
we have decided to focus our efforts on designing a secure
\ac{SECH} 2, and neglect to expound here a
version of \ac{SECH} 1 optimised to its particular
design motivations and requirements.

\subsection{Implementation Notes}
The implementation of \ac{SECH} 2 can be used to run \ac{SECH} 1
by configuring the \varsechlongtermkey{} with a publicly
known value such as the 32 byte null string.