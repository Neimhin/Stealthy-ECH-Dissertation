% \subsection{Design}
% It turns out that \ac{SECH} 1 can be implemented 
% by running \ac{SECH} 2 (which we'll describe below),
% except using a publicly known \varsechlongtermkey{}.

% The basic idea is to use a symmetric key encryption to encrypt the
% inner \ac{SNI} and \ac{ALPN}, and then replace the \var{random}
% and \varlegacysessionid{} fields of the \var{ClientHello}
% with the ciphertext. Since the ciphertext is pseudorandom
% a censor is forced to perform a decryption operation using
% the publicly known key in order to detect \ac{SECH} 1.

% Some of the design choices we lay out for \ac{SECH} 2 are overkill
% for \ac{SECH} 1.
% For instance the acceptance confirmation signal
% could be dropped in the case of \ac{SECH} 1.
% Due to the limited use-cases for \ac{SECH} 1, however,
% we have decided to focus our efforts on designing a secure
% \ac{SECH} 2, and neglect to expound here a
% version of \ac{SECH} 1 optimised to its particular
% design motivations and requirements.

% \subsection{Implementation Notes}
% The implementation of \ac{SECH} 2 can be used to run \ac{SECH} 1
% by configuring the \varsechlongtermkey{} with a publicly
% known value such as the 32 byte null string.