\subsection{\var{HelloRetryRequest} Hijacking}
\label{sec:hrr-hijacking}
If a client sends \var{key\_share}s for cipher suites the server does not support the
server will issue a \ac{HRR} listing supported cipher suites to which the
client responds with a \ac{CH2} and a new \var{key\_share} for one of the supported cipher suites.
The \ac{HRR} hijacking attack involves an attacker constructing a malicious \ac{CH2}
such that the server later encrypts the \ac{tlsC} to the attacker, revealing the inner \ac{SNI}.

More fully, the flow would go as follows: a client sends a legitimate \ac{CHO}, the server successfully
decrypts and accepts the inner servername but replies with
a legitimate \ac{HRR},
an attacker intercepts the \ac{HRR} and constructs
a malicious \ac{CH2} with an attacker-controlled \var{key\_share}
while also constructing
a new \varech{} extension.
The server completes the handshake with the attacker,
but using the inner servername
from the first legitimate \ac{CHO}.

The mitigation for this described in the \ac{ECH} draft is to use the same encryption context for the first and second \ac{CH}s.
That is to say the encapsulated symmetric key \var{enc} is sent only in the first \ac{CHO}
and reused in order to encrypt/decrypt the second \varech{}.
If \ac{ECH} decryption fails for the second \ac{CHO}
then the server aborts.
This prevents forgery of the second \ac{CHO}.

For our approach to \ac{SECH} we cannot copy the \ac{ECH} mitigation.
The first \ac{SECH} payload is hidden in the cover of the \ac{CH}\var{.random}
and \varlegacysessionid{}.
In \ac{TLS} 1.3, however, these fields are repeated identically in the \ac{CH2},
which means we cannot re-do the \ac{SECH} payload encryption for \var{ClientHello2}.
The fields in which we might have placed the repeated \ac{SECH} payload encryption
cannot be changed compared the first \ac{CH}, or else
the handshake will stick out.

For an earlier draft of this dissertation we proposed an \ac{SECH} protocol that was vulnerable to this type of attack.
In that variant the \ac{SECH} synthetic transcript contained
no secret inner random.
This meant if the
server accepted the inner servername and sent a \ac{HRR}, then an attacker would be able to intercept
the \ac{HRR} and respond with an attacker-controlled \ac{CH2}.
In this vulnerable design the server did
not try to decrypt the \ac{SECH} payload from the second \ac{CH},
but rather carried over the inner \ac{SECH} parameters decrypted from the first \ac{CH}.
The attacker controlled \ac{CH2}
would be processed by the client-facing server and forwarded to the secret backend server.
The handshake keys would be derived from the synthetic inner transcript.
Once the attacker receive the \var{Certificate} message
it could repeatedly try to generate the correct handshake keys
by guessing different possible
transcripts of \ac{CHI}.
Since the synthetic inner transcript did not contain an inner random
the attacker could guess correctly with reasonable probability
and thus eventually decrypt the \var{Certificate} message to learn the inner \ac{SNI}.

We can gain some protection
against this attack
by the incorporation on an inner random which prevents the attacker
from guessing the transcript.
However, this mitigation is not as strong as that of the \ac{ECH} approach.
The \ac{ECH} approach provides defense in depth since the attack is mitigated
by the shared \ac{HPKE} context of the two \ac{CH}s {\em as well as}
the inner random.
% It is possible that a more thorough cryptographic analysis will reveal
% the inner random to provide insufficient protection against this attack.

In particular, the \ac{ECH} design prevents the server from ever sending the encrypted
\ac{tlsC} message unless the \ac{CH2} is legitimate.
Relying only on the inner random does not prevent this,
and once the \ac{tlsC} has been sent the remainder of the
attack can be performed offline.
Also, the \var{HRR} hijack means that the secret established
by (\ac{EC})\ac{DHE} is known to the attacker, so the
only thing protecting the handshake traffic secrets
is the secret inner random in the synthetic transcript.
The transcript in the \ac{TLS} 1.3 key schedule is not designed
to provide this type of protection, rather it is designed
to prevent transcript malleability.
% Protection of 

Irritatingly, in order not to stick out from \ac{TLS} 1.3
we need to design the \ac{SECH} protocol such that an attacker
can perform \var{HRR} hijacking  in the same way
they would be able to for \ac{TLS} 1.3.
This means that if an attacker intercepts the \var{HRR}
and injects an attacker-controlled \ac{CH2},
then the server should respond with a regular \ac{TLS} 1.3
\ac{SH} and \ac{tlsEE}.
If instead we used the \ac{CHI} as the transcript
in the key schedule,
then the attacker would
fail to decrypt \ac{tlsEE} which would
reveal to the attacker that non-standard \ac{TLS} was in use.
For this reason it turns out that triggering an \var{HRR}
while running \ac{SECH} is fatal.
If a server needs to reply with a \var{HRR} then the only
stealthy solution (we have found)
is to give up on completing a connection
to the backend server, and instead complete
the connection with the host identified by the outer \ac{SNI}.
This way, if the connection is hijacked by an attacker
the attacker will not learn any secret values of the \ac{CHI}, nor break \ac{PC}.