\chapter{Introduction}


In this dissertation we propose designs
for variants of \ac{SECH} extensions
to the \ac{TLS} 1.3 protocol,
with the goal of offering enhanced
privacy compared to regular \ac{TLS} 1.3.
We report on the development
of implementations and tests of the proposed protocols
in a fork of the OpenSSL library,
and we provide cursory analyses and arguments
attesting to the security of the
proposed protocols.

The \ac{TLS} 1.3 protocol enables secure connections between
a client and server over an insecure medium such as the Internet.
In particular \ac{TLS} aims to provide confidentiality of messages
and authentication of the communicating parties (either server-only authentication
or mutual authentication).
One of the most popular applications of \ac{TLS} is
in the \ac{HTTPS} protocol,
but \ac{TLS} has many use-cases beyond \ac{HTTPS}.

\cite{rfc7258-pervasive-monitoring} declared that \ac{PM} is a ``widespread attack on privacy'', and
that ``the \ac{IETF} will work to mitigate \ac{PM}''.
The \ac{TLS} protocol is very often
deployed in a way that is vulnerable
to \ac{PM}, in particular due to
information leaked in the cleartext \ac{SNI} and
\ac{ALPN} extensions.
The \ac{SNI} is a well-known factor used
by censors to discriminate connections to be blocked.
The \ac{ALPN} also has the potential
to be abused by censors.
Both \ac{PM} and censorship are harmful to Internet users.

The draft \ac{ECH} protocol \citep{esni} is an
extension to \ac{TLS} which allows the \ac{SNI}, \ac{ALPN},
and other values to kept confidential.
However, it is trivial for an on-path attacker
campaign to distinguish between \ac{ECH} enabled \ac{TLS} connections, and regular \ac{TLS} connections,
i.e. \ac{ECH} is not stealthy.
In other words the \ac{ECH} protocol does not offer \ac{PC}.

Many cases of aggressive state-sponsored Internet censorship exist and persist today.
It is sometimes the case that states willing to aggressively censor the Internet
are also willing to employ violence and other forms of coercion/influence to achieve their goals.
The act of trying to circumvent censorship
can attract punishment.
Therefore the fact that \ac{ECH} is not stealthy
is a deterrent to protocol adoption for people suffering such
regimes.
A successful \ac{SECH} protocol would be impossible for
a censor to identify without control of the client or server device.
This guarantee of \ac{PC} would make the protocol more
attractive than \ac{ECH} for a user who
wishes to circumvent censorship but
needs to also hide the fact that they are attempting to evade
detection.

There is a risk that a widespread deployment of \ac{ECH} will trigger powerful censors to block
all \ac{ECH} connections,
resulting in some
degree of splintering of the Internet.

The social and political aspects relating to the
deployment of \ac{ECH}-like protocols are expansive,
but this work is primarily an investigation of a {\em technical method}
which could be used to mitigate or prevent \ac{PM} and censorship
based on the \ac{SNI} and \ac{ALPN}.

Our designs of \ac{SECH} turn out to have several
disadvantages in comparison to \ac{ECH},
arising from the prioritisation of maintaining \ac{PC}.
In particular, under various failure cases our \ac{SECH}
protocols
require more bandwidth and chattiness between the client
and server to maintain stealth.
Also, our designs are less flexible in terms of
the underlying cipher suites that can be used for
the servername encryption.


% [ ] even when censorship is not present argue that pervasive monitoring/lack of privacy are bad for individuals and society

% [ ] acknowledge that privacy is often violated in pursuit of seemingly noble goals: counter-terrorism, criminal investigations

% [ ] slippery slope from counter-terrorism to plain-old spying for illegitimate purposes (political sway, propaganda, commerce etc.)


% \section{ECH}
%The \ac{ECH} extension to \ac{TLS}, under development
%by \cite{esni},
%is a promising proposal
%intended to provide greater privacy for \ac{TLS} users.
%However, \ac{ECH} falls short in one of its
%design goals: ``Don't stick out''.

While the \ac{ECH} proposal provides a mechanism
by which it may in future achieve its goal
of not sticking out (namely the use of the \ac{GREASE} pattern for \ac{ECH}),
it will take time to see if that mechanism
will be deployed to the extent where it becomes effective.
Also, \ac{ECH} is designed in such a way that it
is trivial for an on-path network observer to
detect that one of \ac{ECH} or \ac{GREASE} \ac{ECH}
is being used.
This means that it is equally trivial for a powerful
censor to block deployments of \ac{ECH}.
The purpose of \ac{GREASE} is that it make blocking only true \ac{ECH}, while allowing
\ac{GREASE} \ac{ECH}, difficult for the censor.

% [ ] the ECH extension to TLS looks promising as a privacy enhancement but entails a risk of splintering the internet because of how much it 'sticks out'

The basic idea behind \ac{ECH} is to use \ac{HPKE} to encrypt a so-called \ac{CHI},
with the encrypted \ac{CHI} in a \ac{TLS} extension.
Encrypting with a public key, which can be distributed to the client before the \ac{ECH} handshake is initiated,
means that the secret inner \ac{SNI} and \ac{ALPN} values can be sent in the client's
first flight of messages, whereas the first encrypted messages in a normal \ac{TLS} 1.3 handshake
are sent by the server (i.e. the second flight of messages).
\cite{rfc8744-issues} laid out the \ac{IETF}'s core issues and requirements relating to the
objective of servername encryption,
and the \ac{ECH} protocol appears to achieve most of those goals,
but deprioritises others.

The privacy-benefit offered by \ac{ECH} is relative to a so-called anonymity set.
A surveillance apparatus that observes an \ac{ECH} connection is able to narrow down the possible
values of the inner \ac{SNI} to that anonymity set, but no further.
Therefore, the greater the size of the anonymity set
the greater the privacy-benefit for the user.
If a censor wishes to block access to a particular website \var{banned.com},
but that website is protected under \ac{ECH} relative to an anonymity set,
then the censor is forced to over-block all websites in the anonymity set,
or not block \var{banned.com} at all.

One of the assumptions that steered the design of \ac{ECH} is that
censors will try to avoid over-blocking for various reasons,
in order to avoid harm to economic activity and growth,
or to avoid social unrest etc., and that the incentive
to avoid over-blocking will be stronger than the incentive to block in enough cases.

On the other hand censors might purposefully over-block in order to disincentivise the deployment of new (privacy enhanced/censorship resistant) protocols.
The design of \ac{ECH} unfortunately leaves the ball in the censor's court;
will they over-block or won't they?
An \ac{SECH} protocol which is indistinguishable from
regular \ac{TLS} has the implication that in order
to completely block \ac{SECH} the censor must
block all \ac{TLS} communications,
which would essentially disable the web as we know it today
for affected users.

On the global stage privacy enhancement and censorship resistance are sides of one coin;
lesser privacy entails an increased capacity to censor.

% \section{Motivations for a stealthy variant of ECH}

An \ac{SECH} protocol should
offer what we will call \ac{PC},
which is when a surveillance system cannot distinguish whether or not
a particular protocol is in use.
In other words we treat the act of using a protocol itself as privacy-sensitive
and aim to design a technical solution that can offer \ac{PC}.

%The goal then, is to design a version of \ac{ECH} that from
%the perspective of a surveillance system or active attacker is indistinguishable
%from regular \ac{TLS} 1.3.
The basic mechanism we will leverage to achieve this centres around
the fact that,
from the perspective of someone
who does not possess the decryption key,
strongly encrypted data is very hard\footnote{Very hard in the sense that a huge number of samples are needed to distinguish between a set of ciphertexts and random strings.} to distinguish from
a string of uniformly random bits.
Additionally, there are several fields in the \ac{TLS}
messages consisting of uniformly random bit strings or pseudo-random strings.
By replacing these (pseudo)random fields with encrypted data we can send
stealthy signals without an observer being able to tell whether the signal was sent or not.

In Chapter~2 we expand on the history and properties of
the \ac{TLS} and \ac{ECH} protocols,
discuss Internet censorship and \ac{PM} concepts,
and acknowledge some other viable approaches to
circumvent censorship and surveillance based on the \ac{SNI}.

In Chapter~3 we go into greater detail on how \ac{TLS}
and \ac{ECH} work,
including the underlying cryptographic primitives,
and expand on the properties of encryption that
make ciphertexts difficult to distinguish from random text.
We then comb through the \ac{TLS} 1.3 specification
to evaluate the quality of (pseudo)random fields in the messages
which could be leveraged as cover for the \ac{SECH} payloads.
And finally discuss the threat model for \ac{SECH}
and types of attack that could be leveled against it.

In Chapter~4 we describe our designs for \ac{SECH} in detail
and report on the state of our implementations.
We call versions of \ac{SECH} which don't assume a shared secret \ac{SECH} 1, and have neglected to provide a design for this.
\ac{SECH} 2 assumes an out-of-band
mechanism for establishing a shared symmetric encryption key. \ac{SECH} 3, like \ac{ECH}, is based on \ac{PKC}.

In Chapter~5 we give further justification for the designs presented in
Chapter~4.
In particular we describe attacks against previous drafts of both \ac{ECH}
and our \ac{SECH} designs, and describe how these attacks have been mitigated.
We then discuss security and privacy considerations for the use of our \ac{SECH} protocols.

In Chapter~6 we summarise what we have learned about designing
an \ac{SECH} protocol under the chosen threat model and requirments. We then discuss
various avenues for future work and research related to the project.