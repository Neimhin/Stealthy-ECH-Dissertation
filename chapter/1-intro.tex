\chapter{Introduction}


In this dissertation we propose designs
for variants of \ac{SECH} extensions
to the \ac{TLS} 1.3 protocol,
with the goal of offering enhanced
privacy compared to regular \ac{TLS} 1.3.
We report on the development
of implementations and tests of the proposed protocols
in a fork of the OpenSSL library,
and we provide cursory analyses and arguments
attesting to the security of the
proposed protocols.

The \ac{TLS} 1.3 protocol enables secure connections between
a client and server over an insecure medium such as the internet.
In particular \ac{TLS} aims to provide confidentiality of messages
and authentication of the communicating parties (either server-only authentication
or mutual authentication). One of the most popular applications of \ac{TLS} is
in the \ac{HTTPS} protocol, but \ac{TLS} has many use-cases beyond \ac{HTTPS}.

\cite{rfc7258-pervasive-monitoring} declared that \ac{PM} is a ``widespread attack on privacy'', and
that ``the \ac{IETF} will work to mitigate \ac{PM}''.
The \ac{TLS} protocol is very often
deployed in a way that is vulnerable
to \ac{PM}, in particular due to
information leaked in the cleartext \ac{SNI} and
\ac{ALPN} extensions.
Also, the \ac{SNI} is a well-known factor used
by censors to discriminate connections to be blocked. The \ac{ALPN} also has the potential
to be abused by censors.

Both \ac{PM} and censorship are harmful to Internet users.

Many cases of aggressive state-sponsored Internet censorship exist and persist today.
It is sometimes also the case that states willing to aggressively censor are also willing to employ violence and other forms of coercion to achieve their goals.

% [ ] even when censorship is not present argue that pervasive monitoring/lack of privacy are bad for individuals and society

% [ ] acknowledge that privacy is often violated in pursuit of seemingly noble goals: counter-terrorism, criminal investigations

% [ ] slippery slope from counter-terrorism to plain-old spying for illegitimate purposes (political sway, propaganda, commerce etc.)

This work is an investigation of a technical method
intended to mitigate or prevent \ac{PM} and censorship
based the \ac{SNI} and \ac{ALPN}.

% \section{ECH}
The \ac{ECH} extension to \ac{TLS}, under development
by \cite{esni},
is a promising proposal
intended to provide greater privacy for \ac{TLS} users.
However, \ac{ECH} falls short in one of its
design goals: ``Don't stick out''.

While the \ac{ECH} proposal provides a mechanism
by which it may in future achieve its goal
of not sticking out (namely the use of the \ac{GREASE} pattern for \ac{ECH}),
it will take time to see if that mechanism
will be deployed to the extent where it becomes effective.
Also, \ac{ECH} is designed in such a way that it
is trivial for an on-path network observer to
detect that either \ac{ECH} or \ac{GREASE} \ac{ECH}
is being used.
This means that it is equally trivial for a powerful
censore to block deployments of either \ac{ECH} or \ac{GREASE} \ac{ECH}.

There is a risk that a widespread deployment of \ac{ECH} will trigger powerful censors to block
all \ac{ECH} connections, resulting in some
degree of splintering of the Internet.

% [ ] the ECH extension to TLS looks promising as a privacy enhancement but entails a risk of splintering the internet because of how much it 'sticks out'

The basic idea behind \ac{ECH} is to use \ac{HPKE} to encrypt a so-called \ac{CHI},
with the ciphertext of the \ac{CHI} in a \ac{TLS} extension.
Encrypting with a public key, which can be distributed to the client before the \ac{ECH} handshake is initiated,
means that the secret inner \ac{SNI} and \ac{ALPN} values can be sent in the client's
first flight of messages, whereas the first encrypted messages in a normal \ac{TLS} 1.3 handshake
are sent by the server (i.e. the second flight of messages).
\cite{rfc8744-issues} laid out the \ac{IETF}'s core issues and requirements relating to the
objective of servername encryption,
and the \ac{ECH} protocol appears to achieve most of those goals,
but deprioritises others.

The privacy-benefit offered by \ac{ECH} is relative to a so-called anonymity set.
A surveiller that observes an \ac{ECH} connection is able to narrow down the possible
values of the inner \ac{SNI} to that anonymity set, but no further.
Therefore, the greater the size of the anonymity set
the greater the privacy-benefit for the user.
If a censor wishes to block access to a particular website \var{banned.com},
but that website is protected under \ac{ECH} relative to an anonymity set,
then the censor is forced to over-block the entire anonymity,
or not block \var{banned.com} at all.

One of the assumptions that steered the design of \ac{ECH} is that
censors will try to avoid over-blocking for various reasons,
in order to avoid harm to economic activity and growth,
or to avoid social unrest etc., and that the incentive
to avoid over-blocking will be stronger than the incentive block in enough cases.

On the other hand censors might purposefully over-block in order to disincentivise the deployment of new (privacy enhanced/censorship resistant) protocols.
The design of \ac{ECH} unfortunately leaves the ball in the censor's court;
will they over-block or won't they?

On the global stage privacy enhancement and censorship resistance are sides of one coin,
and we proceed under the tenet that is not appropriate to consider them separately from each other.
Lesser privacy entails an increased capacity to censor.

% \section{Motivations for a stealthy variant of ECH}

One of the motivations for designing a \ac{SECH} protocol is to
offer what we will call \ac{PC},
which is when a surveillance system cannot distinguish whether or not
a particular protocol is in use.
In other words we treat the act of using a protocol itself as privacy-sensitive
and aim to design a technical solution that can offer \ac{PC}.

The goal then, is to design a version of \ac{ECH} that from
the perspective of a surveillance system or active attacker is indistinguishable
from regular \ac{TLS} 1.3.
The basic mechanism we will leverage to achieve this centres around
the fact that (strongly) encrypted data is indistinguishable from
a string of uniformly random bits from the perspective of someone
who does not possess the decryption key.
Additionally, there are several fields in the \ac{TLS}
messages consisting of uniformly random bit strings or pseudo-random strings.
By replacing these (pseudo)random fields with encrypted data we can send
stealthy signals without an observer being able to tell whether the signal was sent or not.

% Introduction to the material covered in the document.

% \section{Style of English}
% \label{sec:StyleOfEnglish}

% Style of English
% An impersonal style keeps the technical factors and ideas to the forefront of the discussion and you in the background. Try to be objective and quantitative in your conclusions. For example, it is not enough to say vaguely “because the compiler was unreliable the code produced was not adequate”. It would be much better to say “because the XYZ compiler produced code which ran 2-3 times slower than PQR (see Table x,y), a fast enough scheduler could not be written using this algorithm”. The second version is more likely to make the reader think the writer knows what he/she is talking about, since it is a lot more authoritative. Also, you will not be able to write the second version without a modicum of thought and effort.

% The following points are couple of {\it Do's \& Dont's} that I have noted down as feedback to reports over the years. The focus of this list is to encourage writers to be specific in writing reports - some of this is motivated by Strunk and White's The Elements of Style~(\cite{strunk}). Regarding reports that are submitted as part of a degree, examiners have to read and mark these reports - make it easy for these examiners to give good marks by following a number of simple points:

% \begin{description}
% 	\item [Acronyms:] Acronyms should be introduced by the words they represent followed by the acronym in capitals enclosed in brackets e.g. "...TCP (Transmission Control Protocol)..." $\Rightarrow$  "... Transmission Control Protocol (TCP)..."
% 	\item [Contractions:] I would generally suggest to avoid contractions such as "I'd", "They've", etc in reports. In some cases, they are ambiguous e.g. "I'd" $\Rightarrow$ "I would" or "I had" and can lead to misunderstandings.
% 	\item [Avoid "do":] Be specific and use specific verbs to describe actions.
% 	\item [Adverbs:] Adverbs and adjectives such as "easily", "generally", etc should be removed because they are unspecific e.g. the statement "can be easily implemented" depends very much on the developer. 
% 	\item [Articles:] "A" and "an" are indefinite articles; they should be used if the subject is unknown. "The" is a definite article; which should be used if a specific subject is referred to. For example, the subject referred to in "allocated by the coordinator" is not determined at the time of writing and so the sentences should be changed to "allocated by a coordinator".
% 	\item [Avoid brackets:] Brackets should not be used to hide sub-sentences, examples or alternatives. The problem with this use of brackets is that it is not specific and keeps the reader guessing the exact meaning that is intended. For example "... system entities (users, networks and services) through ..." should be replaced by "... system entities such as users, networks, and services through ...".
% 	\item [Figures:] Figures and graphs should have sufficient resolution; figures with low resolution appear blurred and require the reader to make assumptions.
% 	\item  [Captions:] Use captions to describe a figure or table to the reader. The reader should not be forced to search through text to find a description of a figure or table. If you do not provide an interpretation of a figure or table, the reader will make up their own interpretation and given Murphy's law, will arrive at the polar opposite of what was intended by the figure or table.
% 	\item [Backgrounds:] Backgrounds of figures and snapshots of screens should be light. Developers often use terminals or development environments with dark backgrounds. Snapshots of these terminals or developments are difficult to read when placed into a report. 
% 	\item [Titles:] Titles of section should never be followed immediately by another title e.g. a title of a chapter should be followed by text describing the content and relevance of the sections of the chapter and could then be followed by the title of the first section of the chapter.
% 	\item [Punctuation:] A statement is concluded with a period; a question with a question mark.  
% 	\item [Spellcheckers:] Use a spellchecker!
% \end{description}


% \section{Figures} 

% The arranging of figures in Latex can lead to spending a lot of time on minor issues e.g. positioning a figure in a specific location on a page, fixing minor issues with an exact size of a figure, etc. Figure~\ref{fig:ImageOfAChick} provides a simple example that demonstrates the use of one of two macros for handling figures, called {\it includefigure}; the other macro,  {\it includescalefigure}, is demonstrated in chapter~\ref{chap:Evaluation}. Figures should always be readable without magnification when printed and the resolution of an image should be sufficient to provide a clear picture when printed.

% \includefigure{fig:ImageOfAChick}{An Image of a chick}{A caption should describe the figure to the reader and explain to the reader the meaning of the figure. A Sub-clause of Murphy's Law: If the interpretation of a figure is left to a reader, the reader will misinterpret the figure, feel insulted or decide to ignore it. Do not leave it up to the reader!}{image.png}


% \section{Structure \& Contents}

% At the end of the introduction, a layout of the structure and the contents of the following chapters should be provided for the reader. The overall goal of all descriptions of contents that follows these descriptions is to prepare the reader. The reader should not be surprised by any content that is being presented and should always know how content that is currently being read fits within an overall dissertation.
