\chapter{Introduction}

[ ] design, implement, test privacy enhancements and censorship resistance improvements to TLS

[ ] pervasive monitoring and censorship harmful to internet users

[ ] many examples of aggressive state-sponsored internet censorship by various means

[ ] often the case that states willing to aggressively censor are also willing to employ violence and other forms of coersion to achieve their goals

[ ] even when censorship is not present argue that pervasive monitoring/lack of privacy are bad for individuals and society

[ ] acknowledge that privacy is often violated in pursuit of seemingly noble goals: counter-terrorism, criminal investigations

[ ] slippery slope from counter-terrorism to plain-old spying for illegitimate purposes (political sway, propaganda, commerce etc.)

[ ] this work primarily an investigation of a technical method of privacy enhancement and censorship circumvention

\section{ECH}
[ ] the ECH extension to TLS looks promising as a privacy enhancement but entails a risk of splintering the internet because of how much it 'sticks out'

[ ] short summary of properties/goals of ECH, don't stick out, anonymity sets, extensibilty, GREASE

[ ] censors might try to avoid 'over-blocking' (encourage economic activity, passify netizens), but on the other hand censors might purposefully over-block in order to disincentivize the deployment of new (privacy enhanced/censorship resistant) protocols

[ ] On the global stage privacy enhancement and censorship resistance are sides of one coin, not appropriate to consider them separate from each other. Worse privacy entails increased capacity to censor.

[ ] privacy is a bulwark against future censorship

\section{Motivations for a stealthy variant of ECH}

% Introduction to the material covered in the document.

% \section{Style of English}
% \label{sec:StyleOfEnglish}

% Style of English
% An impersonal style keeps the technical factors and ideas to the forefront of the discussion and you in the background. Try to be objective and quantitative in your conclusions. For example, it is not enough to say vaguely “because the compiler was unreliable the code produced was not adequate”. It would be much better to say “because the XYZ compiler produced code which ran 2-3 times slower than PQR (see Table x,y), a fast enough scheduler could not be written using this algorithm”. The second version is more likely to make the reader think the writer knows what he/she is talking about, since it is a lot more authoritative. Also, you will not be able to write the second version without a modicum of thought and effort.

% The following points are couple of {\it Do's \& Dont's} that I have noted down as feedback to reports over the years. The focus of this list is to encourage writers to be specific in writing reports - some of this is motivated by Strunk and White's The Elements of Style~(\cite{strunk}). Regarding reports that are submitted as part of a degree, examiners have to read and mark these reports - make it easy for these examiners to give good marks by following a number of simple points:

% \begin{description}
% 	\item [Acronyms:] Acronyms should be introduced by the words they represent followed by the acronym in capitals enclosed in brackets e.g. "...TCP (Transmission Control Protocol)..." $\Rightarrow$  "... Transmission Control Protocol (TCP)..."
% 	\item [Contractions:] I would generally suggest to avoid contractions such as "I'd", "They've", etc in reports. In some cases, they are ambiguous e.g. "I'd" $\Rightarrow$ "I would" or "I had" and can lead to misunderstandings.
% 	\item [Avoid "do":] Be specific and use specific verbs to describe actions.
% 	\item [Adverbs:] Adverbs and adjectives such as "easily", "generally", etc should be removed because they are unspecific e.g. the statement "can be easily implemented" depends very much on the developer. 
% 	\item [Articles:] "A" and "an" are indefinite articles; they should be used if the subject is unknown. "The" is a definite article; which should be used if a specific subject is referred to. For example, the subject referred to in "allocated by the coordinator" is not determined at the time of writing and so the sentences should be changed to "allocated by a coordinator".
% 	\item [Avoid brackets:] Brackets should not be used to hide sub-sentences, examples or alternatives. The problem with this use of brackets is that it is not specific and keeps the reader guessing the exact meaning that is intended. For example "... system entities (users, networks and services) through ..." should be replaced by "... system entities such as users, networks, and services through ...".
% 	\item [Figures:] Figures and graphs should have sufficient resolution; figures with low resolution appear blurred and require the reader to make assumptions.
% 	\item  [Captions:] Use captions to describe a figure or table to the reader. The reader should not be forced to search through text to find a description of a figure or table. If you do not provide an interpretation of a figure or table, the reader will make up their own interpretation and given Murphy's law, will arrive at the polar opposite of what was intended by the figure or table.
% 	\item [Backgrounds:] Backgrounds of figures and snapshots of screens should be light. Developers often use terminals or development environments with dark backgrounds. Snapshots of these terminals or developments are difficult to read when placed into a report. 
% 	\item [Titles:] Titles of section should never be followed immediately by another title e.g. a title of a chapter should be followed by text describing the content and relevance of the sections of the chapter and could then be followed by the title of the first section of the chapter.
% 	\item [Punctuation:] A statement is concluded with a period; a question with a question mark.  
% 	\item [Spellcheckers:] Use a spellchecker!
% \end{description}


% \section{Figures} 

% The arranging of figures in Latex can lead to spending a lot of time on minor issues e.g. positioning a figure in a specific location on a page, fixing minor issues with an exact size of a figure, etc. Figure~\ref{fig:ImageOfAChick} provides a simple example that demonstrates the use of one of two macros for handling figures, called {\it includefigure}; the other macro,  {\it includescalefigure}, is demonstrated in chapter~\ref{chap:Evaluation}. Figures should always be readable without magnification when printed and the resolution of an image should be sufficient to provide a clear picture when printed.

% \includefigure{fig:ImageOfAChick}{An Image of a chick}{A caption should describe the figure to the reader and explain to the reader the meaning of the figure. A Sub-clause of Murphy's Law: If the interpretation of a figure is left to a reader, the reader will misinterpret the figure, feel insulted or decide to ignore it. Do not leave it up to the reader!}{image.png}


% \section{Structure \& Contents}

% At the end of the introduction, a layout of the structure and the contents of the following chapters should be provided for the reader. The overall goal of all descriptions of contents that follows these descriptions is to prepare the reader. The reader should not be surprised by any content that is being presented and should always know how content that is currently being read fits within an overall dissertation.
