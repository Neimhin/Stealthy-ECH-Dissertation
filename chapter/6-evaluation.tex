\chapter{Evaluation}
\label{chap:Evaluation}

\section{General Security and Privacy Considerations for SECH}

\subsection{Protocol Confidentiality}
As mentioned in Section~\ref{bleichenbacher-attack} any good \ac{SECH} protocol should protect the confidentiality of the fact that \ac{SECH} is being attempted and the fact that \ac{SECH} is supported. There are aspects of the protocol design that will protect this information, for instance how the connection continues with regular \ac{TLS} 1.3 whenever \ac{SECH} is aborted.
However, ensuring the confidentiality of this information depends also on
a careful and secure information. For instance, it may be possible for an attacker to somehow with high likelihood guess the library/implementation used for a particular channel. With this information the attacker could then analyse the timing of messages, or the particular alert messages under various circumstances to gain information about whether \ac{SECH} is in use.
The recurring discovery of \ref{bleichenbacher-attack}-style attacks are an attestation to the difficulty of implementing protocols that do not reveal such secret information.

To make matters worse, unlike \ac{RSA} private messages/keys which consist of many bits, the fact of a protocol being used/supported can be represented by a single bit, so protocol confidentiality is broken if an attacker can attain this single bit, meaning a successful attack will likely require much fewer oracle accesses than is the case for \ref{bleichenbacher-attack}. It should not be taken for granted that any given implementation of \ac{SECH} sufficiently protects protocol confidentiality.

% The Evaluation chapter should present a comparison of the work that forms the basis of the dissertation and existing work. At a higher level, it should demonstrate an awareness of the relationship of the dissertation work to the research area that it is based in.

% \section{Experiments}

% In the case where experiments have been carried out, the experimental setup and the values that were defined for the variables need to be presented in a table e.g. table~\ref{tab:experimentsetup}.

% \begin{table}[!h]
% \begin{center}
% 	\begin{tabular}{|l|c|c|} 
% 	\hline
%  	\bf Column 1  & \bf Column 2  & \bf Column 3 \\
%   	\hline
% 	Row 1 & Item 1 & Item 2 \\
% 	Row 2 & Item 1 & Item 2 \\
% 	Row 3 & Item 1 & Item 2 \\
% 	Row 4 & Item 1 & Item 2 \\
% 	\hline
% 	\end{tabular}
% \end{center}
% \caption[Variables of the experiment]{Caption that explains the table to the reader}	
% \label{tab:experimentsetup}
% \end{table}


% \section{Results}

% Figures that present results such as figure~\ref{fig:measurements} need to display descriptions of the axes, the units and scales of the measurements, statistical values, etc. Where measurements were taken from experiments, error bars or confidence intervals need to be provided to give the reader an indication of the spread of the measurements.

% \includescalefigure{fig:measurements}{Measurement of System Wakeups}{Long caption that describes the figure to the reader}{1}{measurements.png}


% \section{Summary}

% Every chapter aside from the first and last chapter should conclude with a summary that presents the outcome of the chapter in a short, accessible form. 