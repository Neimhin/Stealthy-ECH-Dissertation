\chapter{Evaluation}
\label{chap:Evaluation}

\section{General Security and Privacy Considerations for SECH}

\subsection{Protocol Confidentiality}
As mentioned in Section~\ref{bleichenbacher-attack} any good \ac{SECH} protocol should protect the confidentiality of the fact that \ac{SECH} is being attempted and the fact that \ac{SECH} is supported, which we'll call \ac{PC}.
There are two levels of \ac{PC};
connection-level \ac{PC},
and channel-level protocol condifentiality.
With connection-level \ac{PC} an attacker
cannot determine whether a particular connection uses \ac{SECH},
although the attacker may be able to determine that the
client/server pair participating in the connection are
capable of using \ac{SECH}.
With channel-level \ac{PC} the attacker
is neither able to determine whether a particular
connection uses \ac{SECH},
nor whether the client/server pair are capable of \ac{SECH}.

The proposed \ac{SECH} 1 protocol provides neither level of
\ac{PC}. With \ac{SECH} 2 we have aimed
to provide connection-level \ac{PC} in all cases,
and channel-level \ac{PC} for some types of deployment (i.e. where the the \ac{SECH} 2 capability is 
never advertised publicly).

We also aspired to make both levels of \ac{PC} possible
with the \ac{HPKE} based \ac{SECH} 5, however
we imagine the main benefit of the \ac{HPKE}-variant
is that the capability can be advertised and distributed
on the \ac{DNS}, in which channel-level \ac{PC} is forfeited.

In a situation where a user is attempting to circumvent censorship it could be that the censor will respond harshly or violently to detected circumvention attempts.
The risk of \ac{PC} being broken may be
very severe, so reader's should note that the protocols
presented in this dissertation have had nowhere near
the same level of analysis or attention as regular \ac{TLS} or \ac{ECH},
and it should be assumed that the design and implementation
have as yet undiscovered vulnerabilities.

There are aspects of the protocol design that will protect this information, for instance how the connection continues with regular \ac{TLS} 1.3 whenever \ac{SECH} is aborted.
However, ensuring the confidentiality of this information depends also on
a careful and secure information. For instance, it may be possible for an attacker to somehow with high likelihood guess the library/implementation used for a particular channel. With this information the attacker could then analyse the timing of messages, or the particular alert messages under various circumstances to gain information about whether \ac{SECH} is in use.
The recurring discovery of Bleichenbacher-style attacks \citep{ronen2019ninelivesbleichenbacher} are an attestation to the difficulty of implementing protocols that do not reveal such secret information.

To make matters worse, unlike \ac{RSA} private messages/keys which consist of many bits, the fact of a protocol being used/supported can be represented by a single bit, so \ac{PC} is broken if an attacker can attain this single bit, meaning a successful attack against a vulnerable system would likely require much fewer oracle accesses than is the case for the attack described by \cite{bleichenbacher1998chosen}. It should not be taken for granted that any given implementation of \ac{SECH} sufficiently protects \ac{PC}.

\subsection{Non-random \var{random} fields}
The design approach adopted in this work involves
using ciphertext in place of all or part of the 
\var{random} field, whose value in \ac{TLS} is
supposed be securely randomly generated.

In the case of \ac{SECH} 2 the first 12 bytes of the \var{random}
are still random, but the remainder consists of
ciphertext, which is produced as a function of the session key,
the \nonce, the cleartext and the context.
We assume in this work that the total entropy of these
four inputs is greater than that of a 32 octet random string,
and also that the encryption algorithm does a good job of
concentrating that entropy into the ciphertext such that the
combined \var{random} and \varlegacysessionid{} fields contain
greater than 32 bytes of entropy.
A good direction for research into attacks against the \ac{SECH} 2
protocol would debunk these assumptions.

Another possible angle for attack would be to leverage
the relationship between the the \var{random} and its
surrounding context.
This relationship does not exist in \ac{TLS} 1.3
since the \var{random} is generated independently of its context.
It could turn out that the relationship exposes \ac{SECH}
to a new vulnerability, not only to break \ac{PC}
but also the standard guarantees of \ac{TLS} 1.3.
As yet we have not discovered a vulnerability exploiting this
relationship.

% The Evaluation chapter should present a comparison of the work that forms the basis of the dissertation and existing work. At a higher level, it should demonstrate an awareness of the relationship of the dissertation work to the research area that it is based in.

% \section{Experiments}

% In the case where experiments have been carried out, the experimental setup and the values that were defined for the variables need to be presented in a table e.g. table~\ref{tab:experimentsetup}.

% \begin{table}[!h]
% \begin{center}
% 	\begin{tabular}{|l|c|c|} 
% 	\hline
%  	\bf Column 1  & \bf Column 2  & \bf Column 3 \\
%   	\hline
% 	Row 1 & Item 1 & Item 2 \\
% 	Row 2 & Item 1 & Item 2 \\
% 	Row 3 & Item 1 & Item 2 \\
% 	Row 4 & Item 1 & Item 2 \\
% 	\hline
% 	\end{tabular}
% \end{center}
% \caption[Variables of the experiment]{Caption that explains the table to the reader}	
% \label{tab:experimentsetup}
% \end{table}


% \section{Results}

% Figures that present results such as figure~\ref{fig:measurements} need to display descriptions of the axes, the units and scales of the measurements, statistical values, etc. Where measurements were taken from experiments, error bars or confidence intervals need to be provided to give the reader an indication of the spread of the measurements.

% \includescalefigure{fig:measurements}{Measurement of System Wakeups}{Long caption that describes the figure to the reader}{1}{measurements.png}


% \section{Summary}

% Every chapter aside from the first and last chapter should conclude with a summary that presents the outcome of the chapter in a short, accessible form. 