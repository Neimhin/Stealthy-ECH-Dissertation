Notes from Stephen Farrell
- Explain SECH earlier.
- Too much detail on OTP etc. but OK to leave it.
- Make nonce brittleness clearer for SECH 2
- More detail on implementation
- Example transcript with annotations
- Ben Schwartz Compact TLS for Protocol Confidentiality


\subsubsection{How ECH mitigates cut-and-paste attacks}
Here we examine the mechanism used in the current draft of ECH to mitigate cut-and-paste attacks. A cut-and-paste attack against a servername encryption protocol could be used by an attacker to unveil the intended backend server of a connection as described in Section 3.1 of \cite{rfc8744-issues}. The following description of how a cut-and-paste attack could leak a cut-and-paste attack is after \cite{rfc8744-issues}.
\begin{enumerate}
    \item A benign client sends a ClientHello with an encrypted SNI included.
    \item An attacker copies the encrypted SNI into a new ClientHello.
    \item The client-facing/backend servers have no mitigation against the replay of the encrypted SNI and establish a connection between the attacker and backend server.
    \item To conclude the handshake the backend server sends \var{Certificate} and \var{CertificateVerify} messages, revealing the identity of the backend server that the benign client intended to reach. 
\end{enumerate}
\cite{esni} note in Section 10.10.1 that a cut-and-paste attack is not possible against ECH because a server either processes a \var{ClientHelloOuter} or \var{ClientHelloInner} and \var{ClientHelloInner.random} is encrypted. Since \var{ClientHelloInner.random} is encrypted using a public key from an ECHConfig, only the client (who encrypted it) and the possessor of the corresponding private key can view \var{ClientHelloInner.random}. Therefore, if a malicious client tries to replay an \var{encrypted\_client\_hello} extension, it will contain a \var{ClientHelloInner.random} value that has already been used.

To understand why ECH doesn't leak the inner SNI in the case of a cut-and-paste attack, let's look at how the attack would play out:
\begin{enumerate}
    \item A client sends a \var{ClientHelloOuter} with an encrypted \var{ClientHelloInner}.
    \item Attacker cuts and pastes the entire ClientHelloOuter (cannot cut-and-paste just the \var{encrypted\_client\_hello} extension because the encrypted payload is cryptographically bound to the ClientHelloOuter).
    \item The client-facing server successfully decrypts the replayed \var{encrypted\_client\_hello} payload and forwards ClientHelloInner to the backend server. (A sophisticated backend server could reject the repeated ClientHelloInner.random, but that is not sufficient to stop the attack [1]).
    \item The backend server accepts ECH (but the attacker cannot verify acceptance because it doesn't have the plain-text ClientHelloInner necessary for \var{transcript\_ech\_conf}, which is in turn necessary for \var{accept\_confirmation}).
    \item Backend server sends \var{ServerHello}, \var{\{Certificate\}}, \var{\{CertificateVerify\}}.
    \item However, {Certificate} is protected with \var{\shts} which can only be derived if the full ClientHelloInner is known. While much of \var{ClientHelloInner} could be guessed by an attacker with reasonable probability, the \var{ClientHelloInner.\-random} can only be guessed with probability $1/2^{256}$ (32 octets). Also, the \var{\shts} is derived from the (EC)DHE handshake, but whether using a \var{key\_share} or a PSK the attacker cannot complete the DH exchange because it doesn't know the private values associated with \var{key\_share}, nor the \var{key\_share} which is encrypted in the \var{ClientHelloInner} (same for PSK).
\end{enumerate}


In summary, an attacker can trigger a backend server to send an encrypted {Certificate}, but cannot decrypt the {Certificate}.
\section{ The Diffie-Hellman Key Exchange }
The \ac{DH} key exchange protocol allows two parties, Alice and Bob, to establish a secret shared key
while communicating over an insecure channel \cite{diffie-hellman-1976}.
Variants and evolutions of the \ac{DH} key exchange protocol
are fundamental to how \ac{TLS} works.

The original \ac{DH} key exchange was based on the
`apparent difficulty of computing logarithms over a finite field $GF(q)$ with a prime number $q$ of elements' \cite[p. 8]{diffie-hellman-1976}.
A finite field (a.k.a. Galoid field), has a finite number of elements on which addition and multiplication are defined such that an additive inverse exists for all
elements and a multiplicative inverse exists for every nonzero element.
The integers with modular arithmetic mod a prime number give us a finite field.

The \ac{DH} key exchange on finite field $GF(q)$ works as follows.
Alice and Bob each generate a secret independent random number $X_a$ and $X_b$ chosen uniformly from ${1,2,\dots,q-1}$.
Alice and Bob agree on a public value $g$, and compute another public value $Y_i=g^{X_i}\mod q$.
Alice acquires $Y_b$ from Bob over an insecure channel, and vice versa.
% The original \ac{DH} description does not specify how Bob authenticates ownership of $Y_b$.
Alice and Bobs' shared secret is defined as $K_{ab}=g^{X_aX_b}\mod q$.
Alice computes this as $Y_b^{X_a}\mod q$ which is equivalent to $Y_a^{X_b}\mod q$.
Thus, Alice and Bob now each posses the shared secret $K_{ab}$.

When attacking \ac{DH} passively we observe $g$, $q$, $Y_a$, and $Y_b$.
We know that $Y_a=g^{X_a}\mod q$, and therefore that
\begin{equation}
$X_a=\log_g Y_a \mod q$
\end{equation}

\section{ Key Exchange Modes }
There are three key exchange modes defined in RFC 8446:
- (EC)DHE
- PSK-only a.k.a. \var{psk\_ke}
- and PSK with (EC)DHE a.k.a. \var{psk\_dhe\_ke}

(EC)DHE is the default key exchange mode. PSK-based key exchanges can be offered by the client
in the \var{psk\_key\_exchange\_modes} extension in the \var{ClientHello}.

When establishing PSKs via the \var{NewSessionTicket} message the PSK is computed as:
\var{HKDF-Expand-Label(resumption\_master\_secret, "resumption", ticket\_nonce, Hash.length).

Similarly we can define a PSK that is valid for resuming under an SECH protected servername as:
\var{HKDF-Expand-Label(resumption\_master\_secret, "sech res", sech\_ticket\_nonce, Hash.length).

In PSK-only mode the server does not sends neither Certificate the CertificateVerify message,
server authentication is implied by the PSK which is assumed to have been shared securely between
the client and server.
Note, however, that the lack of the ephemeral key sharing in PSK-only mode means that PSK-only does
not provide forward secrecy, i.e. if the PSK is compromised then all connections established using PSK-only and
that PSK lose confidentiality.

A PSK can be used to bootstrap SECH, i.e. to allow a client to connect to a backend SECH server,
because the PSK's identity can be bound either directly or indirectly to the backend servername.
If the PSK is an externally shared PSK then it would be bound directly to the backend servername,
and if the PSK identity was derived in a previous handshake then it is bound to that previous handshake
and implicitly to the inner servername of that handshake.

\section{ Qs }
What's the simplest design of a shared-mode ticket system for SECH that will allow for bootstrapping access?
- client offers SECH
- server accepts SECH and sends NewSessionTicket
- ticket (PSK identity) is cryptographically bound to the session and the inner servername
- on new handshake client sends PSK identity and binder
- server responds with an SECH accept_confirmation and inner servername's Certificate if using (EC)DHE

\section{ Finding Cover }
\subsection{ padding extension }
The padding extensions \var{extension\_data} field `consists of an arbitrary number of zero bytes'.
This padding could be used as cover for SECH by using the length of the \var{extension\_data} as a piece
of data itself. Say we use padding as cover for some bits and set the maximum padding size to 15, then
the size of the padding encodes 4 bits of data:
0  -> 0000
1  -> 0001
2  -> 0010
3  -> 0011
...
15 -> 1111

The tradeoff between transmission size and stealthy bits is terrible here,
a padding extension of size 16 takes up 20 bytes (160 bits),
counting in the extension header, but offers only 4 covert bits.
Even worse, the total covert bits scales logarithmically with the maximum padding size.

The padding extension was introduced to solve issues with buggy implementations, and using
the padding extension for covert bits would require some engineering to ensure that the extension
could fulfill its original purpose.
Also, if the padding extension is used to encode covert bits (which may for instance be part of a cipher text),
then these covert bits will look uniformly random when tracked across sessions, which an attacker
could detect, thus compromising stealth.
